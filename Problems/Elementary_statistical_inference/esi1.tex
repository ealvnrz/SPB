%@ Subject: Elementary Statistical Inference

\addpoints
\question[30] Se sabe que el tiempo de reacción de un humano, en segundos, se puede modelar mediante una distribución normal. Un psiólogo estima que la desviación estándar del tiempo de reacción es $0.05\ [s]$. En un estudio posterior, se toma una muestra aleatoria de 25 personas, la cual resulta tener un tiempo medio de reacción de $0.5\ [s]$. 
\noaddpoints
\begin{parts}
\part[10] Construya un intervalo de confianza del $95\%$ para el tiempo medio de reacción humano.  \part[10] ¿Con que nivel de confianza podemos afirmar que el tiempo medio de reacción está entre $0.5 \pm 0.03\ [s]$?
 \part[10] ¿Es posible suponer que el tiempo medio de reacción difiere de $0.55\ [s]$?. Utilice una confianza del $95\%$. Utilice: $-t_{1-\alpha/2}(n-1)=t_{\alpha/2}(n-1)$ \end{parts}

\begin{solution}

$X:\{$Tiempo de reacción humano en $[s]\}.$ 
\begin{enumerate}[a)]
\item El intervalo de confianza estará dado por: $$IC(\mu)_{(1-\alpha)100\%}=\left[ \overline{X}\pm t_{1-\alpha/2}(n-1)\dfrac{s}{\sqrt{n}}\right]$$ Así, reemplazando con los datos dados por enunciado: $s=0.05\ [s], n=25 ,\overline{X}=0.5\ [s]$, el intervalo de confianza pedido es: $$IC(\mu)_{(1-\alpha)100\%}=\left[0.5\pm 2.064 * 0.01\right]=[0.47936;0.52064]$$

\item El intervalo dado es: $[0.47;0.53]$, por lo que se tiene que:$$0.53-0.47=2*t_{1-\alpha/2}(n-1)\dfrac{s}{\sqrt{n}}$$Luego, reemplazando con los datos dados por enunciado, se tiene que $3\approx t_{1-\alpha/2}(24)$. Buscando el valor más cercano en la tabla de probabilidad t-student se tiene que:$$0.998=1-\alpha/2\Rightarrow \alpha=0.004$$Por lo que el nivel de confianza pedido es $99.6\%$

\item Planteando las hipótesis:
$$H_0: \mu=0.55 \hspace{25pt} H_1: \mu \neq 0.55$$
Nuestro estadístico de prueba está dado por: $t=\dfrac{0.5-0.55}{0.05/\sqrt{25}}=-5$, y rechazaremos la hipótesis nula si $t\leq t_{\alpha/2}(n-1)$ o $t\geq t_{1-\alpha/2}(n-1)$. Luego como $t_{0.975}(24)=2.064 \Rightarrow t_{0.025}(24)=-2.064$. Así, dado que $t\leq -2.064$, rechazamos nuestra hipótesis nula.
\end{enumerate}

\end{solution}
