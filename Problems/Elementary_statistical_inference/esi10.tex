
%@ Subject: Elementary Statistical Inference
\addpoints
\question[20] Los siguientes datos representan las calificaciones de física para una muestra aleatoria de 12 alumnos de primer año de cierta universidad, junto con sus calificaciones de una prueba de aptitud que se les aplicó cuando aún eran alumnos de último año de enseñanza media:

\begin{table}[h!]
\centering
\begin{tabular}{|l|l|l|l|l|l|l|l|l|l|l|l|l|}
\hline Calificación en la prueba & 65 & 50 & 55 & 65 & 55 & 70 & 65 & 70 & 55 & 70 & 50 & 55 \\
\hline Calificación en física & 85 & 74 & 76 & 90 & 85 & 87 & 94 & 98 & 81 & 91 & 76 & 74 \\
\hline
\end{tabular}
\end{table}

Se sabe que:

\begin{align*}
&\sum(\text { Calificación en la prueba })=725 \quad, \quad \sum (\text{ Calificación en la prueba })^2=44.475 \\
&\sum(\text { Calificación en física })=1011 \quad, \quad \sum (\text{ Calificación en física })^2=85.905 \\
\end{align*}
La ecuación de la línea de regresión para predecir la calificación en física a partir de la calificación en la prueba viene dada por:
$$
\hat{y}_i=30,043+0,897 x_i \quad, \quad i=1, \ldots, 12
$$
Además, con esta información, se obtuvo $R^2=0,743$.
\begin{parts}
\part Confeccione la tabla de análisis de varianza (ANOVA).
\part Mediante un análisis de varianza, pruebe la hipótesis de que la calificación en la prueba influye en la calificación en química. (Utilice $\alpha=0,05$)
\end{parts}

\begin{solution}
Sea $Y:$ Calificación en física, $X:$ Calificación en la prueba. Se tiene que el modelo de regresión lineal simple está dado por:

$$Y_i=\beta_0+\beta_1 X_i +\epsilon_i,\quad \epsilon_i\sim N(0,\sigma^2)$$

Luego,
$$SSTO=\sum_{i=1}^{12}y_{i}^{2} - 12 \overline{y}^2 = 85.905 - 12 * (84.25)^2=728.25$$
y,
$$R^2=\dfrac{SSR}{SSTO}\Rightarrow SSTO = 0.743 * 728.25 = 541.08975$$

Luego la tabla ANOVA estará dado por:


\begin{center}
\begin{tabular}{|l|l|l|l|l|}
\hline 
F.V.     & SS        & g.l. & MS        & F       \\
\hline
Regresión & 541.08975 & 1    & 541.08975 & 28.9105 \\
Error     & 187.16025 & 10   & 18.716025 &         \\
\hline
Total     & 728.25    & 11   &           &        \\
\hline
\end{tabular}

\end{center}

El test de hipótesis pedido es:

$$H_0: \beta_1=0 \quad H_1: \beta_1 \neq 0$$

y el estadístico de prueba es:

$$F=\dfrac{MSR}{MSE} \sim F(1,10)$$

Rechazaremos nuestra hipótesis nula si $F > F_{0.95}(1,10)=4.965$ (por tabla). Luego, como $28.9105 > 4.965$ se rechaza $H_0$. Así, la calificación en la prueba influye en la calificación en física con una significancia del $5\%$.


\end{solution}