%@ Subject: Sampling Theory


\question Un auditor muestrea aleatoriamente con reposición 20 cuentas impagadas de una empresa verifica en 12 de ellas la cantidad adeudada y si los documentos respectivos cumplen (1) no cumplen (0) con los procedimientos establecidos. Se tienen la siguiente estructura poblacional:

$$\begin{array}{lccccc}\text { Cuenta } & \text { Cantidad } & \text { Concordancia } & \text { Cuenta } & \text { Cantidad } & \text { Concordancia } \\ 1 & 278 & 1 & 11 & 188 & 0 \\ 2 & 192 & 1 & 12 & 212 & 0 \\ 3 & 310 & 1 & 13 & 92 & 1 \\ 4 & 94 & 0 & 14 & 56 & 1 \\ 5 & 86 & 1 & 15 & 142 & 1 \\ 6 & 335 & 1 & 16 & 37 & 1 \\ 7 & 310 & 0 & 17 & 186 & 0 \\ 8 & 290 & 1 & 18 & 221 & 1 \\ 9 & 221 & 1 & 19 & 229 & 0 \\ 10 & 168 & 1 & 20 & 305 & 1\end{array}$$

Basándose en las 12 cuentas verificadas, estimar la proporción de cuentas cuyos documentos concuerdan, así como el importe medio adeudado, y cuantificar el error cometido.