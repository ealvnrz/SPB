%@ Subject: Sampling Theory

\addpoints

\question[60] En un determinado municipio se desea estudiar el cambio relativa en el valor actual de los bienes inmuebles en los últimos dos años. Se selecciona una muestra aleatoria de tamaño $n=20$ inmuebles de entre los $N=1000$ totales. De los registros oficiales, se obtiene el valor registrado actual para este año (X) y el valor correspondiente de hace dos años (Y), de cada una de las $n=20$ casas incluidas en la muestra. Usando la siguiente información:

\begin{center}
\begin{tabular}{|c|c|c|}
\hline 
Casa & Valor hace 2 años & Valor actual \\ 
\hline 
- & $y_i$ & $x_i$ \\ 
\hline 
1 & 6.7 & 7.1 \\ 
\hline 
2  & 8.2 & 8.4 \\
\hline 
3  & 7.9 & 8.2 \\
\hline 
4  & 6.4 & 6.9 \\
\hline 
5  & 8.3 & 8.4 \\
\hline 
6  & 7.2 & 7.9 \\
\hline 
7  & 6   & 6.5 \\
\hline 
8  & 7.4 & 7.6 \\
\hline 
9  & 8.1 & 8.9 \\
\hline 
10 & 9.3 & 9.9 \\
\hline 
11 & 8.2 & 9.1 \\
\hline 
12 & 6.8 & 7.3 \\
\hline 
13 & 7.4 & 7.8 \\
\hline 
14 & 7.5 & 8.3 \\
\hline 
15 & 8.3 & 8.9 \\
\hline 
16 & 9.1 & 9.6 \\
\hline 
17 & 8.6 & 8.7 \\
\hline 
18 & 7.9 & 8.8 \\
\hline 
19 & 6.3 & 7   \\
\hline 
20 & 8.9 & 9.4 \\
\hline
Total & 154.5 & 165.7 \\ 
\hline 
\end{tabular} 
\end{center}

\noaddpoints
\begin{parts}
\part[20] Estimar $R$, el cambio relativo en el valor actual para los $N=1000$ inmuebles del municipio.
\part[20] Obtenga una cota para el error de estimación del cambio relativo en el valor actual de los inmuebles.
\part[20] Es de especial interés el precio medio de los inmuebles en el municipio en la actualidad. Cuantifique porcentualmente la ganancia en precisión al utilizar una estimación indirecta adecuada con respecto a la estimación directa usada bajo un muestreo aleatorio simple.
\end{parts}
\begin{solution}
El cambio relativo $R$ en el valor actual puede ser estimado mediante un estimador de razón, así:
$$\hat{R}=\dfrac{\hat{X}}{\hat{Y}}=\dfrac{\overline{x}}{\overline{y}}=\dfrac{\sum_{i=1}^{n} x_i}{\sum_{i=1}^{n} y_i}=\dfrac{165.7}{154.5}=1.07$$

Luego, para obtener una cota para el error de estimación debemos obtener $\widehat{\mathbb{V}(\hat{R})}$:

$$\widehat{\mathbb{V}(\hat{R})}=\dfrac{(1-\dfrac{n}{N})}{n(n-1)\overline{Y}^2}\left[ \sum_{i=1}^{n} X_{i}^2 + \hat{R}^2 \sum_{i=1}^{n} Y_{i}^2-2\hat{R} \sum_{i=1}^{n} X_i Y_i \right]\approx 0.00005$$

Así, nuestra cota para el error de estimación estará dado por: $2*\sqrt{\widehat{\mathbb{V}(\hat{R})}}=2*\sqrt{0.00005}\approx 0.01490065$

Finalmente, obtenemos la estimación de la media poblacional utilizando una estimación indirecta de razón:

$$\hat{\overline{X}}=\hat{R}*\overline{Y}=8.26575$$

y su varianza asociada:

$$\widehat{\mathbb{V}(\hat{\overline{X}})}=\overline{Y}^2 * \widehat{\mathbb{V}(\hat{R})}=(7.725)^2*0.00005\approx 0.003312437$$

Y contrastamos con la estimación directa de la media poblacional bajo un muestreo aleatorio simple:
$$\widehat{\mathbb{V}(\overline{X})}=\left(\dfrac{N-n}{N}\right)\dfrac{s^2}{n}=0.04488787$$

Por lo que, la \textbf{ganancia porcentual} corresponde a un $92.62064\%$ con respecto a la obtenida bajo una estimación directa.
\end{solution}