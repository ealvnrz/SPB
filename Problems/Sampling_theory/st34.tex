%@ Subject: Sampling Theory
\addpoints
\question El ingreso nacional de las industrias manufactureras se estimará para el año 1989 a partir de una muestra de 6 de las 19 categorías de industrias que reportaron cifras a principios de ese año. Se conocen los ingresos de las 19 industrias correspondientes a 1980 y el total es de 674 mil millones de dólares. A partir de los datos proporcionados, estime el ingreso nacional total de la manufactura en 1989, y obtenga una cota para el error de estimación. Todas las cifras están en miles de millones de dólares constantes (1982).

\begin{center}
\begin{tabular}{lcc}
\hline Industria & 1980 & 1989 \\
\hline Madera y productos de maderas & 21 & 26 \\
Equipos eléctricos y electrónicos & 63 & 91 \\
Vehículos y equipos de motor & 91 & 47 \\
Productos alimenticios y afines & 60 & 70 \\
Productos de fábrica textil & 70 & 70 \\
Productos químicos y afines & 50 & 50 \\
\hline
\end{tabular}
\end{center}
\noaddpoints
\begin{parts}
\item Encuentre un estimador de razón del ingreso total de 1989 y obtenga una cota para el error de estimación.
\item Encuentre un estimador de regresión del ingreso total de 1989 y obtenga una cota para el error de estimación.
\item Encuentre un estimador de diferencias del ingreso total de 1989 y obtenga una cota para el error de estimación.
\item ¿Cuál de los tres métodos usados es el más apropiado en este caso? ¿Por qué?
\end{parts}