%@ Subject: Sampling Theory

\question Mediante muestreo irrestricto aleatorio se trata de estimar la proporción y el total de aciertos
obtenidos en un juego ilegal en el que se realizan un total de 6000 apuestas. En un ensayo
previo se han obtenido 1/3 de fallos en las apuestas. Se pide:

\begin{parts}
\part Hallar el número de apuestas necesario para que el error de muestreo sea de una décima
al estimar la proporción de aciertos en las apuestas del juego ilegal. Hallar también el
número de apuestas necesario para que el error relativo de muestreo sea del $20\%$ en la
misma estimación.

\part Hallar el número de apuestas necesario para que el error de muestreo sea de 600 unidades
al estimar el total de aciertos en las apuestas con un coeficiente de confianza del $99.7\%$ y
suponiendo muestreo aleatorio simple con reposición. Hallar dicho tamaño en las
condiciones anteriores pero para un error relativo de muestreo del $10\%$.
\end{parts}