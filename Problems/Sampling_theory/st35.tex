%@ Subject: Sampling Theory
\addpoints
\question ¿Están los anestesistas sobrecargado de trabajo y por lo tanto poniendo a los pacientes en riesgo? Esta pregunta fue investigada como parte de una encuesta realizada por la Universidad de Florida. La población de los anestesistas fue estratificado en 3 grupos: anestesistas (casi el $50\%$ de la población), anestesistas residentes (casi $10\%$ de la población) y enfermeras anestesistas (casi el $40\%$ de la población). Las frecuencias de aquellos en cada estrato que pensaban que habían
trabajado sin interrupción más allá de un límite seguro en algún momento durante los últimos 6 meses, se muestran en la tabla adjunta:

\begin{center}
\begin{tabular}{llrr}
\hline Job classification & Worked without break & & \\
\hline Anesthesiologist & beyond safe limit & Frequency & Percentage \\
& No & 417 & $31.4$ \\
Anesthesiology resident & Yes & 913 & $68.7$ \\
& No & 29 & $17.6$ \\
Nurse anesthetist & Yes & 136 & $82.4$ \\
& No & 240 & $21.8$ \\
& Yes & 860 & $78.2$ \\
\hline
\end{tabular}
\end{center}
\noaddpoints
\begin{parts}
\item Estime la proporción de población de quienes piensan que han trabajado más allá de un límite seguro y obtenga una cota para error de estimación.
\item ¿Se diferencian significativamente los anestesiólogos de los residentes en este asunto?
\item ¿Se diferencian significativamente los anestesiólogos de las enfermeras anestesistas en este asunto?
\end{parts}