%@ Subject: Sampling Theory
\addpoints
\question Considere una población de 6 estudiantes. Suponga que sus puntajes son:
\begin{center}
\begin{tabular}{l|cccccc} 
Estudiante & 1 & 2 & 3 & 4 & 5 & 6 \\
\hline Puntaje & 66 & 59 & 70 & 83 & 82 & 71
\end{tabular}
\end{center}
\noaddpoints
\begin{parts}
\item Encuentre la media $\overline{y}_{U}$  y la varianza $S^{2}$ de la población.
\item ¿Cuantos M.A.S. de tamaño 4 son posibles?
\item Haga una lista con los posibles M.A.S. Para cada uno, encuentre su media muestral. Además, obtenga un valor para $\mathbb{V}(\overline{y})$.
\item Considere una estratificación como: Estrato 1: estudiantes $1-3$; Estrato 2: estudiantes $4-6$. ¿Cuántos muestras estratificadas simples de tamaño 4 son posibles en las cuales 2 estudiantes son seleccionados desde cada estrato?
\item Haga una lista con las posibles muestras estratificadas aleatorias. ¿Cuáles de las muestras de c) no pueden ocurrir con el diseño estratificado?
\item Encuentre $\overline{y}_{str}$ para cada muestreo estratificado aleatorio posible. Obtenga $\mathbb{V}(\overline{y}_{str})$ y compárelo con $\mathbb{V}(\overline{y})$.
\end{parts}