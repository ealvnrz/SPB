%@ Subject: Sampling Theory

\question De una población con 33 millones de habitantes se ha obtenido una muestra de $10.000$. En ella, $4.000$ se han clasificado como población activa, y de éstos, 40 se encuentran en situación de desempleo. Se pide:
\begin{parts}
\part Estimar el porcentaje de población activa. Estimar también el número de personas activas que se encuentran en situación de desempleo. Calcular los errores absoluto y relativo de muestreo en ambas estimaciones así como intervalos de confianza con un riesgo del 3 por mil.

\part ¿Cuántas personas de todas las edades sería necesario incluir en una muestra para estimar la tasa de actividad a nivel nacional con un error absoluto $E=0,02$ y una probabilidad del $95 \%$? Del último censo se sabe que en el país hay un $39 \%$ de activos. Contestar a la misma pregunta para cometer un error relativo del $5 \%$.
\end{parts}
