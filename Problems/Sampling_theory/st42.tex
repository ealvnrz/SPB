%@ Subject: Sampling Theory
\addpoints

\question[60] Utilizaremos el conjunto de datos \textbf{syc} contenido en el paquete \textbf{SDaA}, correspondiente a una encuesta a adolescentes y adultos jóvenes en instituciones de detención juvenil en los Estados Unidos (1989). Una pequeña descripción de las variables es presentada en el archivo anexo. Utilizando la columna \textbf{finalwt} como los pesos muestrales:\\

\textit{Fuente: Survey of Youth in Custody. Source: InterUniversity Consortium on Political and Social Research, NCJ-130915 (U.S. Department of Justice, 1989)}

\noaddpoints
\begin{parts}
\part[5] Realice un histograma de la edad de los encuestados en el primer arresto usando los pesos muestrales. ¿Hay alguna diferencia en el histograma al no usar los pesos muestrales?

\part[5] Compare la función de distribución empírica de la edad de los encuestados asumiendo que los datos vienen desde una muestra autoponderada y usando los pesos muestrales. Superponga ambas funciones en un mismo gráfico. ¿Se aprecia alguna diferencia? Justifique.

\part[5] Obtenga el primer cuartil y mediana de la edad de los encuestados usando los pesos muestrales. ¿Existe una diferencia en estas cantidades al no tomar en cuenta los pesos muestreales?

\part[10] Obtenga la media de la edad de los encuestados en el primer arresto usando los pesos muestrales y compare esta cantidad con su equivalente bajo una muestra autoponderada.

\part[15] Estime la media de la edad de los encuestados usando bootstrap (100 replicaciones) y jackknife bajo el mismo diseño muestral. Compare estas cantidades con la estimación directa. ¿Qué estimador es más homogéneo?

\part[10] ¿Existe una diferencia significativa entre la edad de los encuestados hispanos en el primer arresto con la del resto de los detenidos?. Justifique.

\part[10] Obtenga la media de la edad de los encuestados en el primer arrestro que fueron detenidos por cometer un crimen violento.
\end{parts}

\begin{solution}

\end{solution}

