%@ Subject: Sampling Theory

\question La siguiente tabla muestra los puntajes de matemática y estadística de 30 estudiantes del primer año de la Universidad de Botswana. 


\begin{center}
$
\begin{array}{llllllll}
\text{Estudiante} & \text{Matemática} & \text{Estadística} & \text{Estudiante} & \text{Género} & \text{Matemática} & \text{Estadística} & \text{Género} \\
1 & 50 & 53 & 1 & 16 & 36 & 60 & 1 \\
2 & 90 & 81 & 1 & 17 & 71 & 55 & 0 \\
3 & 61 & 85 & 1 & 18 & 72 & 57 & 1 \\
4 & 38 & 68 & 0 & 19 & 35 & 82 & 0 \\
5 & 45 & 73 & 0 & 20 & 62 & 62 & 1 \\
6 & 67 & 62 & 1 & 21 & 81 & 20 & 1 \\
7 & 40 & 34 & 1 & 22 & 74 & 55 & 0 \\
8 & 50 & 49 & 1 & 23 & 88 & 32 & 0 \\
9 & 89 & 59 & 1 & 24 & 55 & 59 & 1 \\
10 & 77 & 84 & 1 & 25 & 69 & 61 & 1 \\
11 & 67 & 23 & 0 & 26 & 32 & 66 & 1 \\
12 & 82 & 73 & 0 & 27 & 37 & 74 & 1 \\
13 & 48 & 61 & 1 & 28 & 66 & 92 & 1 \\
14 & 71 & 37 & 0 & 29 & 34 & 68 & 0 \\
15 & 42 & 35 & 1 & 30 & 35 & 71 & 1
\end{array}
$
0, \text { Mujer; } 1, \text { Hombre. }
\end{center}

\begin{parts}
\part Seleccione un muestra de 10 estudiantes con un M.A.S. y estime la media poblacional de los puntajes de matemática y estadística, y sus respectivas varianzas.

\part Calcule un estimador insesgado para la diferencia entre la media entre los puntajes de matática y estadística junto con su desviación estándar.

\part Calcule un estimador insesgado para el Coeficiente de variación poblacional de los puntajes de matemática.

\part Estime la proporción poblacional de estudiantes mujeres y estime su desviación estándar. Calcule un intervalo de confianza con un nivel del $95\%$ para la proporción de estudiantes hombres de primer año.

\part Calcule un estimador insesgado del puntaje medio de matemática y estadística de los estudiantes hombres y sus respectivas varianzas.

\part Encuentre un intervalo de confianza con un nivel del $90\%$ para la diferencia entre el puntaje medio de matemática y estadística para toda la generación.
\end{parts}