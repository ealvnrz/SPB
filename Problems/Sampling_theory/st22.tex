%@ Subject: Sampling Theory

\question La siguiente tabla entrega la altura de 30 plantas de un experimento agrícola.

\begin{center}
\begin{tabular}{llllll} 
Planta & Altura en cm & Planta & Altura en cm & Planta & Altura en cm \\
\hline 1 & 28 & 11 & 53 & 21 & 33 \\
2 & 99 & 12 & 95 & 22 & 12 \\
3 & 96 & 13 & 34 & 23 & 56 \\
4 & 42 & 14 & 40 & 24 & 56 \\
5 & 58 & 15 & 42 & 25 & 29 \\
6 & 26 & 16 & 17 & 26 & 25 \\
7 & 11 & 17 & 16 & 27 & 25 \\
8 & 82 & 18 & 10 & 28 & 88 \\
9 & 68 & 19 & 35 & 29 & 17 \\
10 & 73 & 20 & 86 & 30 & 29
\end{tabular}

\end{center}
\begin{parts}
\part Escriba todos los posibles muestras sistemáticas lineales (de igual diferencia entre elementos muestreados) de tamaño 5 y calcule el estimador de la media poblacional. Compare el desempeño del muestreo sistemático con el de un muestreo aleatorio simple sin reemplazo base en la misma muestra de tamaño 5.

\part Seleccione una muestra sistemática de tamaño 8 y estime la altura media poblacional de las 30 plantas.

\part Seleccione una muestra sistemática de tamaño 10 y construya un intervalo de confianza del $95\%$ para la altura media de las plantas.

\end{parts}
