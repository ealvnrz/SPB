%@ Subject: Sampling Theory
\addpoints
\question[20] El director de una empresa de calzados desea estimar la longitud media del pie derecho de los hombres mayores de edad en una ciudad. Sea $y$ la variable \textit{Longitud del pie derecho} y $x$ su altura. De un censo pasado, el director sabe que la altura media de los hombres mayores de edad en aquella ciudad es de $168$ cm. Para estimar la longitud del pie, el director organiza una encuesta utilizando un muestreo aleatorio simple sin reemplazo de $100$ hombres adultos. Los resultados son los siguientes:
$$\hat{\overline{X}}=169 \hspace{20pt} \hat{\overline{Y}}=24 \hspace{20pt} s_{x}=10  \hspace{20pt} s_{y}=2  \hspace{20pt} s_{xy}=15$$
Si se sabe que viven $400000$ hombres adultos en la ciudad:
\noaddpoints
\begin{parts}
\part[10] Estime la longitud media del pie derecho de los adultos hombres de la ciudad mediante un estimador de razón y diferencia, y sus respectivas varianzas.
\part[5] Cuantifique la variación en precisión al utilizar las estimaciones indirectas obtenidas con respecto a la estimación directa usada bajo un muestreo aleatorio simple.
\part[5] ¿Que estimador recomendaría usar al director de la empresa? Justifique.
\end{parts}
\begin{solution}
Por definición se tiene que:
$$\hat{\overline{Y}}_{\text{Razón}}=\dfrac{24}{169}*168=23.85799$$
y
$$\hat{\overline{Y}}_{D}=24-169+168=23$$
En donde sus varianzas son:
$$\widehat{\mathbb{V}(\hat{\overline{Y}}_{D})}\approx 0.73981$$
y
$$\widehat{\mathbb{V}(\hat{\overline{Y}}_{\text{Razón}})}\approx 0.01756,$$
respectivamente. Luego, como base tendremos la estimación directa dada por:
$$\hat{\overline{Y}}=24$$
y su varianza asociada:
$$\mathbb{V}(\hat{\overline{Y}})=\left(\dfrac{400000-100}{400000}\right)*\dfrac{4}{100} \approx 0.04$$
Así, las variaciones porcentuales en precisión son $-1750\%$ y $+56.09\%$ aproximadamente, para los estimadores de diferencia y razón, respectivamente. Finalmente, notamos que las estimaciones puntales son bastante similares entre sí, pero mediante la estimación de razón obtenemos la menor varianza, por lo que se tendrá un mayor nivel de precisión.
\end{solution}