%@ Subject: Sampling Theory
\addpoints
\question Las forestales desean estimar la edad media de los árboles en un rodal. Determinar la edad es engorroso, debido a que es necesario contar los anillos de los núcleos de los árboles. Sin embargo, cuanto más viejo es el árbol, mayor es el diámetro y el diámetro es fácil de medir. Las forestales miden el diámetro de los 1132 árboles y encuentran que la media de la población es igual a $10.3$.  Luego, seleccionan 20 árboles al azar para medir la edad.

\begin{center}
\begin{tabular}{ccrccr} 
N° árbol & Diámetro, $x$ & Edad, $y$ & N° árbol & Diámetro, $x$ & Edad, $y$ \\
\hline & & & & & \\
1 & $12.0$ & 125 & 11 & $5.7$ & 61 \\
2 & $11.4$ & 119 & 12 & $8.0$ & 80 \\
3 & $7.9$ & 83 & 13 & $10.3$ & 114 \\
4 & $9.0$ & 85 & 14 & $12.0$ & 147 \\
5 & $10.5$ & 99 & 15 & $9.2$ & 122 \\
6 & $7.9$ & 117 & 16 & $8.5$ & 106 \\
7 & $7.3$ & 69 & 17 & $7.0$ & 82 \\
8 & $10.2$ & 133 & 18 & $10.7$ & 88 \\
9 & $11.7$ & 154 & 19 & $9.3$ & 97 \\
10 & $11.3$ & 168 & 20 & $8.2$ & 99
\end{tabular}
\end{center}
\noaddpoints
\begin{parts}
\item Realice un gráfico de dispersión de $y$ vs $x$.
\item Estime la edad media de la población de los árboles en el rodal utilizando la estimación
dar un error estándar aproximado para su estimación.
\item Repita b) usando el estimador de regresión.
\item Incluya sus estimaciones en los gráficos. ¿Cómo se comparan?
\end{parts}