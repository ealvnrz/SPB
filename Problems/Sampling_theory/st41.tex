%@ Subject: Sampling Theory
\addpoints

\question [10] Si denotamos por $\pi_k$ la probabilidad de inclusión de la observación $k-$ésima, $N$ como el tamaño poblacional y $n$ como el tamaño de muestra fijo. Muestre que:

$$\sum_{k\in U} \pi_k=n$$

\begin{solution}
Consideremos:
$$I_k=\begin{cases}
1 \text{ si } k\in \mathcal{S}\\
0 \text{ en otro caso}
\end{cases}$$
Por lo que la variable $I_k$ tiene distribución Bernoulli, $B(1,\pi_k)$. Así,

$$\mathbb{E}(I_k)=\pi_k \hspace{20pt} \mathbb{V}(I_k)=\pi_k(1-\pi_k)$$

Por lo tanto,

$$\sum_{k\in\mathcal{U}} I_k=n \hspace{20pt}\text{(por definición de }n\text{)}$$

En donde $\mathcal{U}$ hace referencia a la población, luego:
$$\mathbb{E}\left(\sum_{k\in\mathcal{U}}I_k\right)=\mathbb{E}(n)=n$$
Finalmente,

$$\sum_{k\in\mathcal{U}}\pi_k=\sum_{k\in\mathcal{U}}\mathbb{E}(I_k)=n$$
\end{solution}

