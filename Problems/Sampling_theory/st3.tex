%@ Subject: Sampling Theory

\addpoints
\question[20] Dos dentistas A y B hicieron una encuesta para investigar el estado de los dientes de 200 niños. El dentista A seleccionó una muestra aleatoria sin reemplazo de 20 niños y contó el número de dientes con
caries de cada niño, obteniendo los siguientes resultados:

\begin{table}[h!]
\centering
\begin{tabular}{l|lllllllllll}
N° de dientes con caries por niño & 0 & 1 & 2 & 3 & 4 & 5 & 6 & 7 & 8 & 9 & 10 \\ \hline
N° niños                          & 8 & 4 & 2 & 2 & 1 & 1 & 0 & 0 & 0 & 1 & 1 
\end{tabular}
\end{table}

El dentista B, utilizando las mismas técnicas dentales y los resultados obtenidos por el dentista A, examinó a los 200 niños y sólo registró aquellos que \textbf{no tenían caries}, encontrando que 60 niños no tenían dientes dañados.\\

\noaddpoints
\begin{parts}
\part[10] ¿Qué dentista obtiene estimaciones más precisas del número total de dientes con caries en los niños?
\part[10] Realice las estimaciones anteriores mediante intervalos de confianza al 95\%. Comente.
\end{parts}

\begin{solution}
Para el dentista A, la estimación del número total de dientes con caries es:
$$\hat{\tau}=N\overline{x}=200*\dfrac{0*8+1*4+\cdots+10*1}{20}=200*2.1=420$$
y el error de estimación estará dado por:
$$\sqrt{\widehat{\mathbb{V}(\hat{\tau})}}=\sqrt{N^2\left(1-\dfrac{n}{N}\right)\dfrac{\hat{s}^2}{N}}=\sqrt{200^2\left(1-\dfrac{20}{200}\right)\dfrac{8.62}{20}}=123.04$$
Para el dentista B, se considera la subpoblación restante de 140 niños con caries, tras eliminar del total inicial los 60 que no tenian caries. En cuanto a la muestra, hay que eliminar de la distribución inicial a los ocho niños que tienen cero caries (20-8=12). La distribución muestral de frecuencia de esta subpoblación restante está dada por:



\begin{center}
\begin{tabular}{lllllllllll}
N° de dientes con caries por niño & 1 & 2 & 3 & 4 & 5 & 6 & 7 & 8 & 9 & 10 \\ \hline
N° niños                          & 4 & 2 & 2 & 1 & 1 & 0 & 0 & 0 & 1 & 1 
\end{tabular}
\end{center}

Por lo que se tiene que $N_1=140$ y $n_1=12$, luego:

$$\hat{\tau_1}=N_1\overline{x_1}=140*\dfrac{1*4+\cdots+10*1}{12}=149*3.5=490$$
y error de estimación estará dado por:
$$\sqrt{\widehat{\mathbb{V}(\hat{\tau_1})}}=\sqrt{N_{1}^{2}\left(1-\dfrac{n_1}{N_1}\right)\dfrac{\hat{s_1}^2}{N_1}}=\sqrt{140^2\left(1-\dfrac{12}{140}\right)\dfrac{9.545}{12}}=419.370$$

Finalmente, los intervalos de confianza pedidos están dados por:
$$IC(\tau)=\hat{\tau}\pm z_{1-\alpha/2} \sqrt{\widehat{\mathbb{V}(\hat{\tau})}}$$
resultando los intervalos:\\
Dentista A: $IC(\tau)=420 \pm 1.96 * 123.04$\\
Dentista B: $IC(\tau)=490 \pm 1.96 * 419.37$
\end{solution}

