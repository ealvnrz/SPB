%@ Subject: Sampling Theory
\addpoints
\question[20] Desde una lista de 100 hogares, 10 fueron escogidos utilizando un muestreo aleatorio simple sin reemplazo, obteniendo la siguiente muestra.

\begin{center}
$\begin{array}{cccccc}
\text{Hogar} & \text{Representante} & \text{Salario del hogar} & \text{Integrantes} & \text{Educación} & \text{Gasto en transporte} \\
\hline 
1 & \mathrm{H} & 2000 & 4 & \text { Primaria } & 200 \\
2 & \mathrm{H} & 3000 & 2 & \text { Primaria } & 250 \\
3 & \mathrm{M} & 4500 & 5 & \text { Secundaria } & 600 \\
4 & \mathrm{H} & 8000 & 3 & \text { Universitaria } & 500 \\
5 & \mathrm{M} & 2000 & 2 & \text { Primaria } & 100 \\
6 & \mathrm{M} & 5000 & 4 & \text { Secundaria } & 150 \\
7 & \mathrm{H} & 7500 & 5 & \text { Universitaria } & 300 \\
8 & \mathrm{H} & 4000 & 3 & \text { Universitaria } & 250 \\
9 & \mathrm{M} & 5000 & 4 & \text { Secundaria } & 200 \\
10 & \mathrm{M} & 6000 & 2 & \text { Universitaria } & 300 \\
\end{array}$
H, Hombre; M, Mujer
\end{center}
\noaddpoints
\begin{parts}
\part[10] Estime la media del salario por hogar y del gasto en transporte, obtenga una cota para el error de estimación para ambos.
\part[10] Estime la proporción de personas que tienen una educación secundaria o superior, y obtenga una cota para el error de estimación a un $95\%$ de confianza.
\end{parts}
\begin{solution}
La estimación del salario medio por hogar y gasto medio en trasporte estarán dado por:
$$\overline{x}=\dfrac{1}{9}\left(2000+3000+\dots+6000\right)=4700$$
y,
$$\overline{y}=\dfrac{1}{9}\left(200+250\dots+300\right)=285,$$
respectivamente.
Luego, para obtener la cota para error de estimación, calculamos sus $s^2$:
$$s_{x}^{2}=4288889 \hspace{30pt} s_{y}^{2}=23916.67$$
Así, 
$$\widehat{\mathbb{V}(\overline{x})}=\left(1-\dfrac{10}{100}\right)\dfrac{4288889}{10}=386000$$
y,
$$\widehat{\mathbb{V}(\overline{y})}=\left(1-\dfrac{10}{100}\right)\dfrac{23916.67}{10}=2152.5$$
Finalmente, como no nos entregan una confianza especifica utilizamos un cuantil $\approx 2$, por lo que las cotas serán:
$$2*\sqrt{386000}=1242.578$$
y,
$$2*\sqrt{2152.5}=92.79009$$
Para el salario medio por hogar y gasto medio en transporte, respectivamente.\\

Para el segundo item, notamos que 7 de las 10 observaciones corresponden a una educación secundaria o superior, por lo que la estimación puntual de la proporción pedida será:
$$\hat{p}=\dfrac{7}{10}$$
Luego, para obtener la cota para error de estimación, calculamos: 
$$\widehat{\mathbb{V}(\hat{p})}=\left(1-\dfrac{10}{100}\right)\dfrac{0.7*0.3}{9}=0.021$$
Por lo que, una cota para el error de estimación a un $95\%$ estará dado por:
$$1.96*\sqrt{0.021}=2.016$$
\end{solution}