%@ Subject: Sampling Theory
\addpoints
\question[20] Consideramos los salarios anuales (variable $X$) en miles de euros de $500$ trabajadores de una empresa de la cual se obtiene la siguiente distribución de frecuencias:
\begin{table}[h]
\centering
\begin{tabular}{l|l}
$X_i$ & $n_i$ \\ \hline
2     & 100   \\
3     & 80    \\
5     & 200   \\
10    & 30    \\
20    & 30    \\
50    & 30    \\
100   & 20    \\
200   & 10   
\end{tabular}
\end{table}

Se estratifica la población en grupos homogénes de ganancias salariales utilizando como variable de estratificación el propio salario anual mediante el criterio dado por $2\leq X <10$, $10\leq X <100$, $100\leq X \leq 200$.
\noaddpoints
\begin{parts}
\part[10] Realizar las afijacaciones de mínima varianza (afijación óptima) sin reposición de una muestra de tamaño $100$ cuando se estima el salario anual medio.
\part[10] Tras la primera afijación, se decide que del estrato $100\leq X \leq 200$ sólo se obtendrán $30$ observaciones. ¿Como será la nueva afijación óptima?
\end{parts}
\begin{solution}
Para cada uno de los estratos se tiene: 
\begin{table}[h]
\centering
\begin{tabular}{|l|l|l|l|}
\hline
Estrato & $S_h$ & $S_{h}^{2}$ & $N_h$ \\ \hline
I       & 1.32  & 1.75        & 380   \\ \hline
II      & 17.1  & 292.13      & 90    \\ \hline
III     & 47.95 & 2298.85     & 30    \\ \hline
\end{tabular}
\end{table}
Así, usando la fórmula de afijación óptima se tiene: 
$$n_{h}=n\dfrac{N_h S_h}{\sum_{h=1}^{L}N_h S_h} \Rightarrow n_1 \approx 15, n_2 \approx 44, n_3\approx 41$$

Seleccionamos 30 unidades para el estrato 3, por lo que las 70 unidades restantes deben ser repartidas mediante afijación de mínima varianza entre los dos primeros estratos. Así, 

$$n_1\approx17 \quad n_2 \approx 53 \quad n_3 \approx 30$$

\end{solution}