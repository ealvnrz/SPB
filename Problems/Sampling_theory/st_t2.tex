%@ Subject: Sampling Theory

La Encuesta Nacional de Calidad de Vida es parte de la vigilancia (mediciones) de diferentes políticas y acciones del estado que buscan la Protección Social. Su información es de vital importancia para evaluar y formular los planes de prevención y atención para las personas que lo necesitan. Su objetivo principal es disponer de información sistemática, confiable y oportuna acerca de la calidad de vida y de la salud de la población chilena; para el diseño, desarrollo y evaluación de las políticas e intervenciones de salud.\\

La base de datos e información necesaria para realizar este trabajo están disponibles en la página oficial del Departamento de Epidemiología del Ministerio de Salud (\href{http://epi.minsal.cl/bases-de-datos/}{Link}). En particular, analizaremos la \textbf{Encuesta Nacional de Calidad de Vida y Salud 2015-16}. Los archivos a disposición son: Base de datos en formato .sav (SPSS) y el manual de uso de la base de datos, que describe la metodología y detalles técnicos del diseño muestral. Adicionalmente estos archivos serán subimos al classroom del curso.\\

El trabajo final de la asignatura consiste en realizar un informe técnico describiendo una muestra obtenido desde un muestreo complejo. El documento deberá contener:

\begin{questions}

\question Descripción del plan de muestreo.
\question Análisis exploratorio de datos de las variables socio demográficas (edad, género, etcétera).
\question Análisis de las variables creadas.
\question Análisis del ingreso mensual líquido de los hogares por \textbf{macrozona}, \textbf{región} y \textbf{área} (Urbano o Rural).
\question Análisis del sistema de salud de los encuestados por \textbf{macrozona}, \textbf{región} y \textbf{área} (Urbano o Rural).
\question Prevalencia de consumo de tabaco de los encuestados por \textbf{región}.
\end{questions}

El informe deberá ser entregado a lo más el \textbf{25/06} mediante el módulo dispuesto a disposición, en conjunto con los códigos utilizados cualquiera sea el software utilizado. La nota final del informe será determinada según la siguiente rúbrica:\\

\begin{center}
\begin{table}[h!]
\resizebox{\textwidth}{!}{%
\begin{tabular}{|l|l|l|l|l|}
\hline
Criterios                                                           & Puntaje & Sobresaliente (5.5-7)                                                                                                                                      & Suficiente (4.0 - 5.4)                                                                                                                                         & Deficiente ($<4.0$)                                                                                                                        \\ \hline
Estructura                                                          & 1 - 7   & \begin{tabular}[c]{@{}l@{}}El trabajo presenta claramente todos los \\ contenidos solicitados.\end{tabular}                                        & \begin{tabular}[c]{@{}l@{}}El trabajo presenta claramente la mayoría de\\  los contenidos solicitados.\end{tabular}                                & \begin{tabular}[c]{@{}l@{}}El trabajo presenta menos de la mitad\\ de los contenidos solicitados.\end{tabular}                   \\ \hline
\begin{tabular}[c]{@{}l@{}}Metodología \\ y resultados\end{tabular} & 1 - 7   & \begin{tabular}[c]{@{}l@{}}La metodología usada es la correcta \\ y los resultados son presentados claramente.\end{tabular}                        & \begin{tabular}[c]{@{}l@{}}La metodología usada es la correcta y\\  los resultados son presentados claramente\\  con errores menores.\end{tabular} & \begin{tabular}[c]{@{}l@{}}La metodología usada es incorrecta\\  y los resultados son erróneos.\end{tabular}                     \\ \hline
Conclusiones                                                        & 1 -7    & \begin{tabular}[c]{@{}l@{}}Las conclusiones son presentadas claramente\\  y demuestran una clara comprensión \\ del material teórico.\end{tabular} & \begin{tabular}[c]{@{}l@{}}Las conclusiones son presentadas claramente \\ y demuestran cierta comprensión\\  del material teórico.\end{tabular}    & \begin{tabular}[c]{@{}l@{}}Las conclusiones son presentadas y \\ no demuestran comprensión \\ del material teórico.\end{tabular} \\ \hline
\end{tabular}%
}
\end{table} 
\end{center}