%@ Subject: Sampling Theory
\addpoints
\question[25] En un estudio sociológico, una muestra sistemática 1 cada 50 fue obtenida desde los registros municipales para determinar el número total de familias que arriendan sus viviendas en determinada comuna. Sea $y_i=1$ si la familia en el $i-$ésimo hogar arrienda y sea $y_i=0$ en caso contrario. Se sabe que existen 15200 hogares en la comuna. Los resultados de la muestra pueden ser resumidos en:
$$\sum_{i=1}^{304} y_i=88$$
\noaddpoints
\begin{parts}
\part[15] Estime el número total de familias que arriendan sus viviendas y obtenga un intervalo de confianza del $95\%$ para esta cantidad. Justifique su respuesta.
\part[10] ¿Es posible afirmar que la proporción de familias que arriendan su hogar es superior al $30\%$? 
\end{parts}
\begin{solution}
La estimación del total estará dado por:
$$\hat{\tau}=N\hat{p}=15200*\dfrac{88}{304}=4400$$
Luego, la varianza del total estará dada por:
$$\widehat{\mathbb{V}(N\widehat{p})}=N^2\mathbb{V}(\widehat{p})$$
en donde se tiene que:
$$\widehat{\mathbb{V}(\widehat{p})}=\left(1-\dfrac{n}{N}\right)\dfrac{\widehat{p}(1-\widehat{p})}{n-1}=0.0006652313$$
Por lo que un intervalo del $95\%$ de confianza estará dado por:
$$[4400 \pm 1.96 * 15200 * 0.02579208]=[3631.602;5168.398]$$ 
Para el segundo item, debemos plantear un test de hipótesis unilateral de la forma:
$$H_0: p \leq 0.3 \hspace{20pt} H_1: p> 0.3$$
Así, el estadístico de prueba bajo $H_0$ ES:
$$E=\dfrac{\widehat{p}-p_0}{\sqrt{\widehat{\mathbb{V}(\widehat{p})}}}\sim N(0,1)$$
reemplazando,
$$E=-0.408122$$
Usando $\alpha=0.05$, comparamos nuestro estadístico de prueba con el cuantil:
$$Z_{1-\alpha}\approx 1.65$$
Finalmente, como $E \ngtr Z_{1-\alpha}$ no rechazamos nuestra hipótesis nula, por lo que no existe evidencia estadística suficiente para afirmar que la proporción de familias que arriendan su hogar es superior al $30\%$.
\end{solution}