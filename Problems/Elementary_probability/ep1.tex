%@ Subject: Elementary Probability
\addpoints
\question[15] Una persona esta interesada en invertir su dinero en acciones en el mercado bursátil nacional. Estudios estadísticos indican que las preferencias por las distintas acciones están representadas por las del tipo A y tipo B. Además, el 45\% de referencias son por las acciones del tipo A. Si la acción es de tipo A, la probabilidad de tener una rentabilidad positiva es de 0.7. Si la acción es de tipo B, la probabilidad de tener una rentabilidad positiva es de 0.6.
\noaddpoints
\begin{parts}
\part[3] Defina sucesos  Identique las probabilidades.
\part[6] ?`Cu\'al es la probabilidad de tener una rentabilidad positiva?
\part[6] Si la rentabilidad es negativa, ?`Cu\'al es la probabilidad que no se haya invertido en acciones del tipo A?
\end{parts}

\begin{solution}
Eventos:
\begin{center}
$A=\{$ La persona invierte en acciones de tipo A$\}$\\$B=\{$ La persona invierte en acciones de tipo B$\}$\\$R=\{$ Se obtiene rentabilidad positiva tras invertir$\}$\\$R^c=\{$ Se obtiene rentabilidad negativa tras invertir$\}$
\end{center}
\begin{align*}
\mathbb{P}(R|A)&=0.7 \Rightarrow \mathbb{P}(A\cap R)=0.315\\\mathbb{P}(R|B)&=0.6 \Rightarrow \mathbb{P}(B\cap R)=0.33
\end{align*}
\begin{align*}
\mathbb{P}(R)&=\mathbb{P}(A)*\mathbb{P}(R|A)+\mathbb{P}(B)*\mathbb{P}(R/B)\\&= 0.45*0.7+0.55*0.6\\&= 0.645 
\end{align*}
Se sabe que,
\begin{align*}
\mathbb{P}(R|B)=0.6\Rightarrow \mathbb{P}(R^c|B)=0.4 =\dfrac{\mathbb{P}(R^c \cap B)}{\mathbb{P}(B)}\Rightarrow \mathbb{P}(B\cap R^c)=0.4*\mathbb{P}(B)=0.4*0.55=0.22 
\end{align*}
Del item b) sabemos que $\mathbb{P}(R)=0,645 \Rightarrow \mathbb{P}(R^c)=0.355$. Luego, reemplazando se obtiene lo pedido.
\begin{align*}
\mathbb{P}(A^c|R^c)=\mathbb{P}(B|R^c)=\dfrac{\mathbb{P}(B\cap R^c)}{\mathbb{P}(R^c)}=\dfrac{0.22}{0.355}\approx 0,62 
\end{align*}
\end{solution}
