%@ Subject: Elementary Probability
\addpoints

\question[10] Los empleados de una empresa constructora están divididos en tres divisiones: administración, operación y ventas. La siguiente tabla indica el número de empleados en cada división clasificados por género:

\begin{center}
\begin{tabular}{lccc}
           & Mujer & Hombre & Totales  \\
Administración & 20 & 30    & 50   \\
Operación & 60 & 140    & 200   \\
Ventas & 100 & 50    & 150     
\end{tabular}
\end{center}
Si se escoge un empleado al azar:\\

\noaddpoints
\begin{parts}
\part[2] ¿Cuál es la probabilidad que el empleado sea mujer? 
\part[2] ¿Cuál es la probabilidad que el empleado sea hombre?
\part[2] ¿Cuál es la probabilidad que el empleado trabaje en operación?
\part[2] ¿Cuál es la probabilidad que el empleado sea mujer y que trabaje en ventas?
\part[2] ¿Cuál es la probabilidad que el empleado sea hombre y trabaje en administración?

\end{parts}

\begin{solution}
La probabilidad de que el empleado sea mujer es: $\mathbb{P}(M)= 180/400$

La probabilidad de que el empleado sea hombre es: $\mathbb{P}(H)= 220/400$

La probabilidad de que el empleado trabaje en operación es: $\mathbb{P}(O)= 200/400$

La probabilidad de que el empleado sea mujer y trabaje en ventas es: $\mathbb{P}(M \cap V)= 100/400$

La probabilidad de que el empleado sea mujer y trabaje en ventas es: $\mathbb{P}(H \cap A)= 30/400$
\end{solution}