%@ Subject: Elementary Probability
\addpoints

\question[20] Los cinturones de seguridad utilizados en la fabricación de aviones son ligeramente prensados 
 para que queden se cierren lo suficiente y así evitar que se aflojen debido a vibraciones. Suponga que el $95\%$ de todos los cinturones de seguridad pasan una inspección inicial. Del $5\%$ que fracasa, el $20\%$ están tan defectuosos que deben ser desechados. Los cinturones restantes se envían a reparación, de los cuales el $40\%$ no pueden ser arreglados y deben ser descartados. El otro $60\%$ de estos cinturones de seguridad son reparados y posteriormente pasan la inspección.
\noaddpoints

\begin{parts}
\part[5] ¿Cual es la probabilidad de que un cinturón de seguridad futuro pase la inspección, ya sea inicialmente o después de ser reparado? 
\part[5] Dado que un cinturón de seguridad pasó la inspección. ¿Cuál es la probabilidad de que haya pasado la inspección inicial y no haya sido necesaria una reparación?
\end{parts}

\begin{solution}
Podemos visualizar el proceso mediante un diagrama de árbol de la siguiente manera:
\begin{center}
% Set the overall layout of the tree
\tikzstyle{level 1}=[level distance=3.5cm, sibling distance=3.5cm]
\tikzstyle{level 2}=[level distance=3.5cm, sibling distance=3cm]

% Define styles for bags and leafs
\tikzstyle{bag} = [text width=4em, text centered]
\tikzstyle{end} = [circle, minimum width=3pt,fill, inner sep=0pt]

% The sloped option gives rotated edge labels. Personally
% I find sloped labels a bit difficult to read. Remove the sloped options
% to get horizontal labels. 
\begin{tikzpicture}[grow=right, sloped]

\node[bag] {$\cdot$}
	child{ 
	child{     node[end, label=right:
                    {$0.01$}] {}
                edge from parent
				node[above] {}
                node[below]  {$20\%$}      
             }     
    child {       
        child {
                node[end, label=right:
                    {$0.16$}] {}
                edge from parent
                node[above] {}
                node[below]  {$40\%$}
            }
            child {
                node[end, label=right:
                    {$0.024$}] {}
                edge from parent
                node[above] {}
                node[below]  {$60\%$}
            }
        edge from parent         
            node[above] {}
            node[below]  {$80\%$}
    }
     edge from parent
     node[below]  {$5\%$}    }              
    child{ node[end, label=right:
                    {$0.95$}] {}
    edge from parent
                node[above] {}
                node[below]  {$95\%$}
		};
\end{tikzpicture}
\end{center}
Si definimos los eventos:
\begin{align*}
I:&\{\text{El cinturón de seguridad pasa la inspección}\} \\
II:&\{\text{El cinturón de seguridad pasa la inspección inicialmente}\} \\
III:&\{\text{El cinturón de seguridad pasa la inspección tras reparación}\} \\
R:&\{\text{El cinturón de seguridad pasa a reparación}\}\\
AR:&\{\text{El cinturón de seguridad es corregido tras la reparación}\}
\end{align*}
Entonces, la primera probabilidad pedida puede ser expresada como:
\begin{align*}
\mathbb{P}(I)&=\mathbb{P}( II \cup III)\\
&= \mathbb{P}(II) + \mathbb{P}( II^c \cap R \cap AR)\\
&= 0.95 + 0.05*0.8*0.6\\
&= 0.974
\end{align*}
La segunda probabilidad pedida puede ser expresada como:
\begin{align*}
\mathbb{P}(\text{ No necesite reparación }| \ I \ )= \dfrac{\mathbb{P}(II)}{\mathbb{P}(I)}=\dfrac{0.95}{0.974}=0.9754
\end{align*}
\end{solution}