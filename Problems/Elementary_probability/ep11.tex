%@ Subject: Elementary Probability

\addpoints
\question[10] Considere la función:
$$f(x)
\begin{cases}
\dfrac{c}{(1+2x)^3}\hspace{10pt} \text{si } 0\leq x \leq \infty\\
0 \hspace{10pt} \text{en otro caso}
\end{cases}
$$
\noaddpoints
\begin{parts}
\part[5] Encuentre el valor de la constante $c$ para que $f(x)$ sea una función de densidad.
\part[5]  Obtenga la función de distribución $F_{X}(x)=\mathbb{P}(X\leq x)$.
\end{parts}
\begin{solution}
Para encontrar el valor de la constante c, notamos que $\int_{0}^{\infty} f(x)=1$. Así,
\begin{align*}
\int_{0}^{\infty} f(x)&= c \int_{0}^{\infty} \dfrac{1}{(1+2x)^3} dx \hspace{15pt} (u=2x+1)\\
&= \dfrac{c}{2} \int_{1}^{\infty} \dfrac{1}{u^3} du \\
&= \lim_{b\rightarrow\infty} \left( -\dfrac{c}{4u^2} \bigg\vert_{u=1}^{b}\right) \\
&= \dfrac{c}{4}
\end{align*}
Así, $c=4$. Luego, para obtener la función de distribución, usando su definición se tiene:
\begin{align*}
F(x)&=\int_{-\infty}^{x} \dfrac{4}{(1+2t)^3} dt= \int_{0}^{x} \dfrac{4}{(1+2t)^3} dt \\
&= \left( -\dfrac{1}{(1+2t)^2} \bigg\vert_{t=0}^{x}\right)\\
&= -\dfrac{1}{(1+2x)^2} + 1\\
&= \dfrac{4x(x+1)}{(2x+1)^2} \hspace{10pt} \text{si } 0\leq x \leq \infty
\end{align*}
\end{solution}