%@ Subject: Elementary Probability

\addpoints
\question[10] Un paciente de cáncer está siendo tratado con una combinación de tres fármacos. Se observa que cuando
se utilizan simultáneamente, la probabilidad de que dos de los tres fármacos se inactiven es de $1/3$, resultando que sólo
uno de ellos permanece activo frente al tumor. La efectividad de cada fármaco, con respecto a producir una remisión del tumor, es diferente.
El fármaco A se ha mostrado efectivo en un $50\%$ de los casos; el fármaco B, en un $75\%$, y el  fármaco C, en un $60\%$.
Asumiendo que la enfermedad remite en el paciente:
\noaddpoints
\begin{parts}
    \part[2] Defina sucesos e identifique probabilidades.
    \part[8] ¿Cuál es la probabilidad de que el responsable de la remisión del tumor sea el fármaco B?
\end{parts}
\begin{solution}
    Se tienen los eventos:
    \begin{align*}
        R:& \text{ Remisión del tumor}\\
        A:& \text{ Fármaco A no se inactiva}\\
        B:& \text{ Fármaco B no se inactiva}\\
        C:& \text{ Fármaco C no se inactiva}
    \end{align*}
Así, se tiene que:
\begin{align*}
    \mathbb{P}(A)&=\mathbb{P}(B)=\mathbb{P}(C)=1/3\\
    \mathbb{P}(R|A)&=0.5\\
    \mathbb{P}(R|B)&=0.75\\
    \mathbb{P}(R|C)&=0.6
\end{align*}
Así, se tiene que la probabilidad pedida está dada por:
$$\mathbb{P}(B|R)=\dfrac{\mathbb{P}(R|B)\mathbb{P}(B)}{\mathbb{P}(R|A)\mathbb{P}(A)+\mathbb{P}(R|B)\mathbb{P}(B)+\mathbb{P}(R|C)\mathbb{P}(C)}=\dfrac{0.75}{0.5+0.75+0.6}=0.4054$$
\end{solution}
