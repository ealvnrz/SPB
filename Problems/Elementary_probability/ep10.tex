%@ Subject: Elementary Probability

\addpoints
\question Una variable aleatoria $X$ tiene por función de masa de probabilidad:
$$f(x)=
\begin{cases}
cx \hspace{10pt} \text{si } x=1,2,\dots,n\\
0 \hspace{10pt} \text{en otro caso}
\end{cases}
$$
\noaddpoints
\begin{parts}
\part Encuentre el valor de la constante $c$.
\part Obtenga la función de distribución. 
\end{parts}
\begin{solution}
Para obtener el valor de la constante $c$ basta notar que $\sum_{x=1}^{n} cx = 1$, por lo que:
$$c\sum_{x=1}^{n}x=1 \Rightarrow c=\dfrac{2}{n(n+1)}$$

Luego, para obtener la función de distribución, por definición se tiene:
$$\sum_{x=1}^{n} c x = \dfrac{2}{n(n+1)} \dfrac{x(x+1)}{2}= \dfrac{x(x+1)}{n(n+1)}$$
Así, la función de distribución estará dada por:
$$F(x)=\dfrac{x(x+1)}{n(n+1)}\hspace{10pt} \text{para } x=1,2,\dots,n$$
\end{solution}
