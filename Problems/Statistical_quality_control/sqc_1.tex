%@ Subject: Statistical Quality Control

\addpoints

\question[47] En el área de postventa de una reconocida marca deportiva se ha detectado un gran número de reclamos asociados a una zapatilla de \textit{running} que lanzaron recientemente. El principal motivo de los reclamos radica en el peso del producto, lo cual provoca problemas para en su uso. Se sabe que el peso objetivo se estableció en 346 gr con una tolerencia de $\pm 5\%$. La empresa cuenta con dos plantas de manufactura, por lo que se tiene acceso a las muestras de lotes producidos en ambas plantas. La siguiente tabla ilustra los resultados obtenidos:

\begin{table}[h!]
\centering
\begin{tabular}{|c|c|c|c|c|c|c|c|c|c|c|}
\cline { 2 - 11 } \multicolumn{1}{c|}{} & \multicolumn{8}{|c|}{ Planta $^{\circ} \mathbf{1}$} & \multicolumn{2}{c|}{ Planta $^{\circ} \mathbf{2}$} \\
\hline $\mathbf{N}^{\circ}$ & Dato 1 & Dato 2 & Dato 3 & Dato 4 & Dato 5 & Dato 6 & Promedio & Rango & Promedio & Rango \\
\hline 1 & 346 & 349 & 345 & 348 & 346 & 348 & 347,00 & 4 & 345,17 & 2 \\
\hline 2 & 340 & 347 & 345 & 346 & 347 & 347 & 345,33 & 7 & 346,33 & 4 \\
\hline 3 & 347 & 348 & 345 & 346 & 347 & 347 & 346,67 & 3 & 344,67 & 4 \\
\hline 4 & 349 & 347 & 346 & 345 & 346 & 346 & 346,50 & 4 & 346,33 & 3 \\
\hline 5 & 346 & 344 & 343 & 346 & 345 & 345 & 344,83 & 3 & 345,5 & 4 \\
\hline 6 & 344 & 348 & 346 & 347 & 344 & 345 & & & 346,5 & 3 \\
\hline 7 & 346 & 347 & 345 & 345 & 343 & 344 & & & 347,33 & 3 \\
\hline 8 & 348 & 349 & 346 & 347 & 347 & 344 & & & 345,83 & 4 \\
\hline 9 & 343 & 345 & 346 & 346 & 347 & 349 & 346,00 & 6 & 346 & 6 \\
\hline 10 & 346 & 347 & 348 & 345 & 345 & 346 & 346,17 & 3 & 346 & 3 \\
\hline 11 & 344 & 344 & 346 & 346 & 345 & 346 & 345,17 & 2 & 346,33 & 3 \\
\hline 12 & 346 & 346 & 349 & 347 & 346 & 347 & 346,83 & 3 & 346 & 4 \\
\hline 13 & 346 & 346 & 345 & 344 & 343 & 345 & 344,83 & 3 & 346,33 & 4 \\
\hline 14 & 345 & 343 & 347 & 346 & 347 & 347 & 345,83 & 4 & 345 & 6 \\
\hline 15 & 347 & 346 & 343 & 348 & 346 & 347 & 346,17 & 5 & 345,5 & 9 \\
\hline
\end{tabular}
\end{table}

Debido a un problema en la digitación del operario encargado del proceso, existe una ausencia de datos en la planta N°1, por que lo que le solicitan que ud. lo solucione. Utilizando los datos entregados:

\noaddpoints
\begin{parts}
\part[2] Obtenga la información faltante de la Planta N°1.
\part[20] ¿Se encuentran las plantas bajo control estadístico? Justifique.
\part[5] Estime la capacidad del proceso para ambas planta. ¿Cuál de las plantas es más \textit{hábil}?.
\part[10] Obtenga los diagramas de control para $R$ en ambas plantas. Comente.
\part[5] ¿Existe evidencia estadística suficiente para aseverar que los pesos medios de las zapatillas manufacturas en las plantas son diferentes? Asuma homocedasticidad y que las condiciones del teorema de límite central aplican.
\part[5] Como encargado de control de calidad de la empresa se le pide entregar un reporte sobre los resultados encontrados. ¿Qué recomendación daría en aquel informe?.
\end{parts}