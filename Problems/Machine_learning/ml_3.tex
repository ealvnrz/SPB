%@ Subject: Machine learning

\addpoints
\question Suponga que se tiene un conjunto de datos con 5 predictores: $X_1:$ Notas promedio, $X_2:$ Coeficiente intelectual, $X_3:$ Nivel educacional (1 para universidad, 0 para enseñanza media), $X_4:$ Interacción entre las notas promedio y el C.I., $X_5:$ Interacción entre las notas promedios y el nivel educacional. La variable respuesta es el salario inicial tras graduarse (en miles de dólares). Suponga que se usa un método de mínimos cuadrados para ajustar un modelo de regresión múltiple, obteniendo $\hat{\beta}_0=50,\hat{\beta}_1=20,\hat{\beta}_2=0.07,\hat{\beta}_3=35,\hat{\beta}_4=0.01,\hat{\beta}_5=-10$.

\noaddpoints
\begin{parts}
\part ¿Qué afirmación es correcta? Justifique
\begin{enumerate}[i.]
\item Para un valor fijo del C.I. y notas promedio, graduados de educación media ganan más, en promedio, que graduados universitarios.
\item Para un valor fijo del C.I. y notas promedio, graduados universitarios ganan más, en promedio, que graduados de enseñanza media.
\item Para un valor fijo del C.I. y notas promedio, graduados de enseñanza media ganan más, en promedio, que los graduados universitarios si es que tienen notas promedio suficientemente altas.
\item Para un valor fijo del C.I. y notas promedio, graduados universitarios ganan más, en promedio, que graduados de enseñanza media si es que tienen notas promedio suficientemente altas.
\end{enumerate}
\part Predecir el salario de un graduado universitario con C.I. 110 y notas medio 4.
\part Verdadero o Falso: Debido a que el coeficiente para la interacción entre notas promedio y C.I. es muy chico, existe poca evidencia de interacción entre estas variables. Justifique su respuesta
\end{parts}