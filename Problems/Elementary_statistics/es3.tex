%@ Subject: Elementary Statistics
\addpoints
\question[15]  Una empresa dedicada a ensamblar circuitos integrados, desea analizar la calidad de los transistores que son comprados a un proveedor determinado. Para ello, se seleccionó una muestra aleatoria de 22 dispositivos y se realizaron ensayos para medir la temperatura máxima de unión. La siguiente tabla presenta los resultados obtenidos:
\begin{table}[h!]
\centering
\begin{tabular}{|c|c|c|}
\hline
Temperatura máxima de unión ($^{\circ}$C) & Frecuencia Absoluta & Frecuencia absoluta acumulada \\ \hline
$[80-110[$                       & 3                   & 3                             \\ \hline
$[110-140[$                      & 4                   & 7 (**)                        \\ \hline
$[140-170[$                      & 6                   & 13                            \\ \hline
$[170-200[$                      & 9 (*)               & 22                            \\ \hline
\end{tabular}
\end{table}
\noaddpoints
\begin{parts}
\part[5] Confeccione un histograma. Comente.
\part[5] Interprete (*) y (**) en el contexto del problema.
\part[5] Registros de una muestra de similares condiciones, seleccionada hace dos años, indican que la temperatura máxima de unión promedio y desviación estándar es de 136 ($^{\circ}$C) y 8 ($^{\circ}$C), respectivamente. Al contrastar los resultados muestrales de ambos conjuntos de datos, ¿Cuál es más homogéneo?
\end{parts}

\begin{solution}
\begin{enumerate}[a)]
\item Histograma y Comentario 
\item 
\begin{itemize}
\item (*): Cantidad de dispositivos de la muestra aleatoria cuya T$^\circ$ máxima de unión en $[^\circ C]$ está entre 170 y 200. 
\item (**):  Cantidad de dispositivos de la muestra aleatoria cuya T$^\circ$ máxima de unión en $[^\circ C]$ es a lo más 140. 
\end{itemize} 
\item Sea $Y:\{$ T$^\circ$ máxima de unión en $[^\circ C]$ de la muestra aleatoria de hace dos años$\}$. Por enunciado sabemos que:
\begin{align*}
\overline{Y}=136 [^\circ C]\hspace{25pt} S_X=8 [^\circ C]
\end{align*}
Sea además, $X:\{$ T$^\circ$ máxima de unión en $[^\circ C]$ de la muestra aleatoria actual $\}$. De donde:
\begin{align*}
\overline{X}=  153.6364 [^\circ C]\hspace{25pt} \sqrt{S_{Y}^2}=\sqrt{1069.5}=32.7032
\end{align*}
\end{enumerate}
Luego, los coeficientes de variación respectivos son:
\begin{align*}
CV_X=\dfrac{32.7032}{153.6364}\approx 0.2128 \hspace{25pt} CV_Y=\dfrac{8}{136}\approx 0.0588
\end{align*}
Por lo que el conjuntos de datos de hace dos años es más homogéneo debido a que su coeficiente de variación es menor. 
\end{solution}