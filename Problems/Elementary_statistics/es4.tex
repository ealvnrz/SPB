%@ Subject: Elementary Statistics
\addpoints
\question[15] El siguiente diagrama de tallo y hoja representa la distribuci\'on de los puntajes en dos pruebas aplicadas sobre los mismos estudiantes. Utilice estos datos para responder las preguntas que aparecen a continuaci\'on:
 \\
\begin{center}
  \begin{tabular}{r|c|l}
  
  \multicolumn{1}{c|}{Prueba 1}& &\multicolumn{1}{c}{Prueba 2} \\
  &2&5\\
  5&3&0\\
  6&4&349\\
  &5&222345589\\
  855322&6&1134789\\
  998744311100&7&4\\
  977430&8&358\\
  &9&8\\ 

  \end{tabular}
  \\\
  \vspace{10pt}
\end{center}


\noaddpoints
\begin{parts}
\part[4] Identifique y clasifique la variable en estudio.
\part[5] Indique una medida de tendencia central adecuada para la distribución de los puntajes de cada una de las pruebas. Calcule e interprete.
\part[6] El 12\% superior recibe un reconocimiento, ¿Desde que puntaje en cada prueba se entregó el reconocimiento?
\end{parts}

\begin{solution}
\begin{enumerate}[a)]
\item Puntaje Prueba 1 $:\{$ Variable cuantitativa discreta en escala intervalar$\}$. \\
Puntaje Prueba 2 $:\{$ Variable cuantitativa discreta en escala intervalar$\}$.  \\

\item En el caso de la prueba 1, debido a la presencia de datos extremos, la medida de tendencia central más adecuada es la mediana:  $$Me=7.2$$ \\
En el caso de la prueba 2, debido a la simetría de los datos, la medida de tendencia central más adecuada es la media: $$\overline{X}\approx 6.007692$$ 
\item Para el cálculo del $12\%$ superior es necesario calcular el $P_{88}$ para ambas pruebas, pues representa el valor mínimo para ser considerado dentro del $12\%$ superior. Así, Usando la fórmula:
$$P_i=X_{\left(\dfrac{i(n+1)}{100}\right)}$$
En donde la posición de los percentiles es 23.76 pues son 26 puntajes en ambas pruebas. Luego, lo percentiles están dados por:
\begin{center}
Prueba 1: $P_{88}= 8.55$ \\
Prueba 2: $P_{88}= 8.4$ 
\end{center}
\end{enumerate}
\end{solution}