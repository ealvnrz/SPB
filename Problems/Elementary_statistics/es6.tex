%@ Subject: Elementary Statistics

\addpoints
\question[12] 

Los siguientes datos corresponden a las cantidades máximas de emisión diarias de óxido de azufre (en toneladas) registradas según planta de emisión, en cierta zona industrial:
\begin{table}[h!]
    \centering
    \begin{tabular}{c|c|c|c} 
        Cantidad de óxido(ton.) & Planta A & Planta B & Planta C \\
        \hline $05-10$ & 50 & 40 & 20 \\
        $10-15$ & 30 & 30 & 40 \\
        $15-20$ & 60 & 0 & 70 \\
        $20-25$ & 20 & 10 & 15 \\
        $25-30$ & 40 & 20 & 5
        \end{tabular}
\end{table}

\noaddpoints

\begin{parts}
    \part[4] Indentifique y clasifique las variables en estudio.
    \part[4] Entre las plantas B y C. ¿Cuál presenta mayor variabilidad relativa según la cantidad de óxido de azufre emitido?
    \part[4] ¿Qué porcentaje de las emisiones producidas por la planta C, supera las 28 toneladas?
\end{parts}

\begin{solution}
    Las variables en estudio son:
    \begin{itemize}
        \item Planta de emisión: Variable discreta en escala nominal
        \item Cantidad de óxido en toneladas: Variable continua en escala de razón
    \end{itemize} 
Por tabla se tiene que para la media aritmética para las plantas B y C son:
\begin{align*}
    \overline{x}_B&=\dfrac{40*7.5+\dots+20*27.5}{100}=14.5\\
    \overline{x}_C&=\dfrac{20*7.5+\dots+5*27.5}{100}=15.67\\
\end{align*}
Y sus varianzas respectivas están dadas por:
\begin{align*}
    S_{B}^{2}&=\dfrac{40*7.5^2+\dots+20*27.5^2}{100}-(14.5)^2=61\\
    S_{C}^{2}&=\dfrac{20*7.5^2+\dots+5*27.5^2}{100}-(15.67)^2=22.37
\end{align*}
Luego, el coeficiente de variación por planta será:
\begin{align*}
    CV_B&=\dfrac{\sqrt{61}}{14.5}=0.5386\\
    CV_C&=\dfrac{\sqrt{47.3941}}{15.67}=0.3
\end{align*}
Así, la planta B presenta mayor variabiidad segun la cantidad de óxido de azufre emitido con respecto a la planta C. Finalmente, se tiene:

\begin{align*}
    P_p&=L_{i-1}+\dfrac{\dfrac{p*n}{100}-N_{i-1}}{n_i}*a_i\\
    28 &=25+\dfrac{\dfrac{p*150}{100}-145}{5}*5 \Rightarrow p=98.66\%
\end{align*}
Luego, $1-p=(100-98.66)\%=1.34$.
\end{solution}