%@ Subject: Elementary Probability

\question[15] El nivel de llenado de unas botellas de bebidas gaseosas tiene una distribuci\'on normal con media 2 [litros] y desviaci\'on est\'andar de 0,06 [litros]. Si las botellas contienen menos de 1,9 [litros], la empresa corre el riesgo de recibir una multa por parte de la entidad encargada de fiscalizar este tipo de productos.\\Por otro lado, si las botellas tienen un contenido mayor a 2,1 [litros], se genera el efecto no deseado de derramar parte del l\'iquido al momento de abrirlas.\noaddpoints\begin{parts}\part[2] Defina la variable bajo estudio.\part[4] Si se selecciona una botella de la producci\'on al azar, ?`C\'ual es la probabilidad de que la empresa corra el riesgo de ser multada?\part[4] Si se selecciona una botella al azar, ?`Cu\'al es la probabilidad de que una botella pueda provocar un derrame?\part[5] Si se obtiene una muestra aleatoria de 30 botellas desde la l\'inea de llenado, ?` Cu\'al es la probabilidad de que haya m\'as de 2 botellas que puedan provocar un derrame al abrirlas?\end{parts}

\begin{solution}
Sea $X:\{$ Nivel de llenado de unas botellas de bebidas gaseosas $\}$ \\$$X\sim N(2,0.06^2) $$\begin{align*}\mathbb{P}(X<1.9)&=\mathbb{P}\left( \dfrac{X-2}{0.06} < \dfrac{1.9 -2}{0.06}\right) \\&=\mathbb{P}\left(Z < -1.\overline{6}\right) \hspace{15pt} Z \sim N(0,1)\\&=0.048\end{align*}\begin{align*}\mathbb{P}((X>2.1)&=1-\mathbb{P}(X\leq 2.1)\\&= 1-\mathbb{P}\left(\dfrac{X-2}{0.06} < \dfrac{2.1-2}{0.06}\right)\hspace{10pt}&\\&= 1- \mathbb{P}\left(Z<-1.\overline{6}\right) Z \sim N(0,1) &\\&= 1- 0.952 &\\&= 0.048 &\end{align*}Sea $Y:\{$ N\'umero de botellas cuyos niveles de llenado es mayor a 2.1$\}$$$Y\sim Bin(30,0.048) $$Luego,\begin{align*}\mathbb{P}(Y>2)&=\mathbb{P}(Y\geq 3) \\&= 1-\mathbb{P}(Y\leq 2)\hspace{15pt} &\\&= 1- 0.8122 \\&=0.1878&\end{align*}
\end{solution}
